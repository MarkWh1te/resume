% !TEX TS-program = xelatex
% !TEX encoding = UTF-8 Unicode
% !Mode:: "TeX:UTF-8"

\documentclass{resume}
\usepackage{zh_CN-Adobefonts_external} % Simplified Chinese Support using external fonts (./fonts/zh_CN-Adobe/)
%\usepackage{zh_CN-Adobefonts_internal} % Simplified Chinese Support using system fonts
\usepackage{linespacing_fix} % disable extra space before next section
\usepackage{cite}

\begin{document}
\pagenumbering{gobble} % suppress displaying page number

\name{马天骏}

\basicInfo{
  \email{iamwh1temark@gmail.com} \textperiodcentered\ 
  \phone{(+86) 18817511974} \textperiodcentered\ }
  %% \linkedin[billryan8]{https://www.linkedin.com/in/billryan8}}

\section{\faTruck\ 个人项目}
\textbf{股票时间序列聚簇分析及可视化输出} http://stock.markwh1te.org

对沪深两市股票的日级前复权股价数据进行了时间序列聚类并可视化输出
\begin{itemize}
  \item 通过对股票形态的聚簇分析快速找到,哪些股票涨跌最明显
  \item用户可以选取不同时间段和行业的股票进行聚簇分析
  \item 对k-means 和DTW算法做了优化,并用Golang重新实现,提升了近10倍运行效率
\end{itemize}
\textbf{个人博客} http://markwh1te.org

从15年开始维护的个人博客,记录了自己的技术成长

\section{\faUsers\ 工作经历}
\datedsubsection{\textbf{中国金融在线(NASDAQ:JRJC)} }{2016.8 -- 至今}
\role{数据挖掘工程师}

灵犀智投项目
\begin{itemize}
  \item 负责股票量化数据回测指标的特征选取和算法实现
  \item 2005到2017年的历史股票交易数据数据清理和数据转化
\end{itemize}


\datedsubsection{\textbf{上海润客投资有限公司}}{2015.6 -- 2016.8}
\role{数据工程师}

极客理财师APP 
\begin{itemize}
\item 
  \item负责基于近4万的金融交流群聊天记录的文本分析, 生成关于产品的热度周期图 
  \item 实现理财产品的推荐系统
\end{itemize}
\end{onehalfspacing}

\section{\faHeartO\ 获奖情况}
\datedline{\textit{第二名},金融界工程师大赛2.0}{2017.9}
\datedline{\textit{优秀奖},上海海事大学数据建模比赛}{2013.10}

\section{\faGraduationCap\  教育背景}

\datedsubsection{\textbf{Neural Networks and Deep Learning by deeplearning.ai},Coursera}{2017.8 -- 至今}
\datedsubsection{\textbf{Big Data Analysis with Scala and Spark},Coursera}{2017.3 -- 2017.4}
\datedsubsection{\textbf{Machine Learning by Stanford University},Coursera}{2016.12 -- 2017.3}

\datedsubsection{\textbf{上海海事大学}, 上海}{2011.9 -- 2015.6}
\textit{学士}\ 

\section{\faCogs\ 技术清单}
% increase linespacing [parsep=0.5ex]
\begin{itemize}[parsep=0.5ex]
  \item 编程语言:C++,Python,Scala,Golang
  \item 机器学习及可视化框架:pandas,numpy,sklearn,seaborn,matiplotlib
  \item 深度学习框架:tensorflow
  \item 开发工具:jupyter notebook,vim,VScode 
  \item 开发环境:Ubuntu,Centos,OSX
\end{itemize}


\section{\faInfo\ 语言}
% increase linespacing [parsep=0.5ex]
\begin{itemize}[parsep=0.5ex]
  \item GRE 300
  \item CET-6 500
\end{itemize}

%% Reference
%\newpage
%\bibliographystyle{IEEETran}
%\bibliography{mycite}
\end{document}
